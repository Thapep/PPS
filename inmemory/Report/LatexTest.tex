\documentclass[12pt,twocolumn]{IEEEtran}

\title{In-Memory Processing: A look into "A Logic-In-Memory Computer", by Harold S. Stone, IEEE 1970}
\date{2020-2021 Fall Semester}
\author{Peppas Athanasios, Iliadis Thrasyvoulos-Fivos}

\begin{document}
  \maketitle

\begin{abstract}
In this work we take a critical look to a "A Logic-In-Memory Computer" and try to evaluate the foundations that were laid on processing in-memory architectures. We pinpoint the persistent ideas from the 70s up to today in the work as well as touch on some inaccuracies and overlooked problems.
\end{abstract}

\section{Introduction}
\subsection{Paper Overview}
 The goal of the article written by Harold S. Stone is bringing into consideration some options on embedding logic-in-memory arrays inside conventional computers of the time. Using a contemporary computer, the IBM 360/85, as an example the paper describes a way to enhance the performance of the cache memory by arming it with some processing capabilities. 
\subsection{Processing-in-Memory}
From the time of the article's writing to today Processing-In-Memory (or In-Memory Processing, PIM)  is a developing field. Generally in computer science, Processing-In-Memory is a group of technologies for the processing of data stored in an in-memory database. Modern PIM architectures rely on the use of RAM because of its faster data accessing times and the emerging demands in Business Intelligence, but older systems implemented such systems based on disk storage and other slower hardware. 

The problem that PIM technologies are trying to address is the reducing of movement of data for computation. Modern architectures are designed with little consideration on data movement optimization. The result of this is data access being a key bottleneck, especially with emergin data-heavy applications ,as well as energy consumption becoming a limiter. With data movement being very costly in terms of bandwidth, these design trends are forcing the field towards other more aware in terms of data movement architectures, one of which is the In-Memory-Processing model.\cite{PAPER:1}

In Section II we show the main points of "A Logic-In-Memory Computer" and give an overview of the key ideas that emerge from it. Section III is where we evaluate the strengths and weaknesses of the original work, while in Section IV we discuss problems that were overlooked and give our recomendations on work to be done.

\section{Paper Overview}
Overview of the sections of the paper

\section{Evaluation}

\section{Further Problems}

\bibliography{myBib}
\bibliographystyle{ieeetr}

\end{document}